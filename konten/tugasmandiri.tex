\section{Tugas Mandiri}
		\begin{enumerate}
		\item Ubah matriks berikut menjadi matriks identitas menggunakan transformasi matriks
		$$
		\begin{array}{llllllllll} 
			a. & 	A=\left[
			\begin{array}{rr} 
				4 & 1\\
				-1 &1 
			\end{array} 
			\right]
			& b. & B=	\left[
			\begin{array}{rr} 
				-1 & 2\\
				3  & 4 
			\end{array} 
			\right]
			& c. & C=	\left[
			\begin{array}{rrr} 
				1 & 2 & -1\\
				2 & -1 & 3\\
				4 & -2 & 1 
			\end{array} 
			\right]
			& d. &  D=\left[
			\begin{array}{rrr} 
				2 & 4 & 3\\
				3 & 1 & 0\\
				-2 &1 & -2 
			\end{array} 
			\right]
			\\
		\end{array} 
		$$
		\item Hitung invers dari matriks berikut menggunakan transformasi matriks%2
		$$
				\begin{array}{llllllllll} 
			a. & 	A=\left[
			\begin{array}{rr} 
				4 & 1\\
				-1 &1 
			\end{array} 
			\right]
			& b. & B=	\left[
			\begin{array}{rr} 
				-1 & 2\\
				3  & 4 
			\end{array} 
			\right]
			& c. & C=	\left[
			\begin{array}{rrr} 
				1 & 2 & -1\\
				2 & -1 & 3\\
				4 & -2 & 1 
			\end{array} 
			\right]
			& d. &  D=\left[
			\begin{array}{rrr} 
				2 & 4 & 3\\
				3 & 1 & 0\\
				-2 &1 & -2 
			\end{array} 
			\right]
			\\
		\end{array} 
		$$
		\item Selesaikan sistem linier berikut dengan menggunakan transformasi matriks\\ %3
				$$
		\begin{array}{lclc} 
			a. & 
			\begin{array}{rcr} 
				4x_1 + x_2&=& 11 \\
				-x_1 + x_2&=& 1 
			\end{array} ~~~~~~~~~~~~
			& b. & 
			\begin{array}{rcr} 
				-x_1 + 2x_2&=& -1 \\
				3x_1 + 4x_2&=& 13 
			\end{array} \\
		\\
			c. & \begin{array}{rcr} 
				x_1 + 2x_2  - x_3 &=& 4 \\
				2x_1 - x_2  + 3x_3 &=& 3  \\
				4x_1 - 2x_2  + x_3 &=& 1 
			\end{array} ~~~~~~~~~~~~
			& d. &  \begin{array}{lcr} 
				2x_1 + 4x_2  + 3x_3 &=& 13 \\
				3x_1 + x_2  &=& 10  \\
				-2x_1 + x_2  - 2x_3 &=& -7 
			\end{array}\\
			\\
		\end{array}
		$$
		\textbf{\textit{Jawaban:}} a. $(2,3)$, b.$(3,1)$, c.$(1,2,1)$, d. $(3,1,1)$
		\end{enumerate}