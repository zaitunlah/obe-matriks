\section{Invers Matriks}
		\paragraph*{} Misalkan matriks $A$ memiliki invers $A^{-1}$, pada proses mendapatkan invers tersebut maka transformasi matriks yang telah dijelaskan sebelumnya akan digunakan untuk mendapatkan invers matriks dengan bentuk 
		$$A~ \vdots~  I~~~~ \longrightarrow ~~~~I ~\vdots~ A^{-1}$$
		dengan menuliskan matriks identitas di bagian kanan matriks $A$ kemudian operasi baris dilakukan bersama dengan matriks identitas.
		\subsection{Matriks $2\times 2$}
		Misalkan diketahui matriks 
		\begin{align*}
		A = \left[
		\begin{array}{rr} 
			2 & 1 \\
			3 & -1 
		\end{array} 
		\right]
	\end{align*}
maka untuk menentukan invers matriks $A$ adalah dengan menuliskan matriks identitas di sebelah kanan
 	\begin{align*}
 	\begin{array}{r|r} 
 		 \left[
 		\begin{array}{rr} 
 			2 & 1 \\
 			3 & -1 
 		\end{array} 
 		\right]  &  \left[
 		\begin{array}{rr} 
 			1 & 0 \\
 			0 & 1 
 		\end{array} 
 		\right]
 	\end{array}
 \end{align*}
Langkah 1: $\textbf{b}^*_1= \textbf{b}_1 + \textbf{b}_2$ \\
\begin{align*}
	\begin{array}{r|r} 
		\left[
		\begin{array}{rr} 
			5 & 0 \\
			3 & -1 
		\end{array} 
		\right]  &  \left[
		\begin{array}{rr} 
			1 & 1 \\
			0 & 1 
		\end{array} 
		\right]
	\end{array}
\end{align*}
Langkah 2 : $\textbf{b}^*_2= 5\textbf{b}_2 - 3\textbf{b}_1$ \\
\begin{align*}
	\begin{array}{r|r} 
		\left[
		\begin{array}{rr} 
			5 & 0 \\
			0 & -5 
		\end{array} 
		\right]  &  \left[
		\begin{array}{rr} 
			1 & 1 \\
			-3 & 2 
		\end{array} 
		\right]
	\end{array}
\end{align*}
Langkah 3 : $\textbf{b}^*_1= \dfrac{1}{5}\textbf{b}_1 $ dan $\textbf{b}^*_2= -\dfrac{1}{5}\textbf{b}_2$ \\
\begin{align*}
	\begin{array}{r|r} 
		\left[
		\begin{array}{rr} 
			1 & 0 \\
			0 & 1 
		\end{array} 
		\right]  &  \left[
		\begin{array}{rr} 
			1/5 & 1/5 \\
			3/5 & -2/5 
		\end{array} 
		\right]
	\end{array}
\end{align*}
dari hasil di atas, tujuan mendapatkan invers matriks  tercapai sesuai $I ~\vdots~ A^{-1}$ sehingga $$A^{-1}=\left[
\begin{array}{rr} 
	1/5 & 1/5 \\
	3/5 & -2/5 
\end{array} 
\right]$$
\subsection{Matriks $3\times 3$}
Serupa dengan proses sebelumnya, untuk ukuran matriks yang lebih besar metode transformasi akan sangat membantu untuk mendapatkan invers matriks. Misalkan matriks berordo $3\times3$
\begin{align*}
	\begin{array}{r|r} 
		\left[
	\begin{array}{rrr} 
		-2 & 1 & 2\\
		2 & 3 & 1\\
		-2 &1 & 3 
	\end{array} 
	\right]  & \left[
	\begin{array}{rrr} 
		1 & 0 & 0\\
		0 & 1 & 0\\
		0 &0 & 1 
	\end{array} 
	\right]
	\end{array}
\end{align*}
Langkah 1: $\textbf{b}^*_1= \textbf{b}_1 - 2\textbf{b}_2~~,~~\textbf{b}^*_2= 3\textbf{b}_2 - \textbf{b}_3$ \\
\begin{align*}
	\begin{array}{r|r} 
		\left[
		\begin{array}{rrr} 
			-6 & -5 & 0\\
			8 & 8 & 0\\
			-2 &1 & 3 
		\end{array} 
		\right]  & \left[
		\begin{array}{rrr} 
			1 & -2 & 0\\
			0 & 3 & -1\\
			0 &0 & 1 
		\end{array} 
		\right]
	\end{array}
\end{align*}
Langkah 2 : $\textbf{b}^*_1= 8\textbf{b}_1 + 5\textbf{b}_2~~,~~\textbf{b}^*_3= 8\textbf{b}_3 - \textbf{b}_2$ \\
\begin{align*}
	\begin{array}{r|r} 
		\left[
		\begin{array}{rrr} 
			-8 & 0 & 0\\
			8 & 8 & 0\\
			-24 &0 & 24 
		\end{array} 
		\right]  & \left[
		\begin{array}{rrr} 
			8 & -1 & -5\\
			0 & 3 & -1\\
			0 &-3 & 9 
		\end{array} 
		\right]
	\end{array}
\end{align*}
Langkah 3 : $\textbf{b}^*_3= \dfrac{1}{3}\textbf{b}_3$ \\
\begin{align*}
	\begin{array}{r|r} 
		\left[
		\begin{array}{rrr} 
			-8 & 0 & 0\\
			8 & 8 & 0\\
			-8 &0 & 8 
		\end{array} 
		\right]  & \left[
		\begin{array}{rrr} 
			8 & -1 & -5\\
			0 & 3 & -1\\
			0 &-1 & 3 
		\end{array} 
		\right]
	\end{array}
\end{align*}
Langkah 4 : $\textbf{b}^*_2= \textbf{b}_2 + \textbf{b}_1~~,~~\textbf{b}^*_3= \textbf{b}_3 - \textbf{b}_1$ \\
\begin{align*}
	\begin{array}{r|r} 
		\left[
		\begin{array}{rrr} 
			-8 & 0 & 0\\
			0 & 8 & 0\\
			0 &0 & 8 
		\end{array} 
		\right]  & \left[
		\begin{array}{rrr} 
			8 & -1 & -5\\
			8 & 2 & -6\\
			-8 &0 & 8 
		\end{array} 
		\right]
	\end{array}
\end{align*}
Langkah 5 : $\textbf{b}^*_1= -\dfrac{1}{8}\textbf{b}_1~~,~~\textbf{b}^*_2= \dfrac{1}{8}\textbf{b}_2~~,~~\textbf{b}^*_3= \dfrac{1}{8}\textbf{b}_3$ \\
\begin{align*}
	\begin{array}{r|r} 
		\left[
		\begin{array}{rrr} 
			1 & 0 & 0\\
			0 & 1 & 0\\
			0 &0 & 1 
		\end{array} 
		\right]  & \left[
		\begin{array}{rrr} 
			-1 & 1/8 & 5/8\\
			1 & 1/4 & -3/4\\
			-1 &0 & 1 
		\end{array} 
		\right]
	\end{array}
\end{align*}
perlu namun tidak harus yaitu saat melakukan operasi kalau bisa tidak menggunakan bilangan pecahan untuk mengantisipasi kesalahan perhitungan yang kadang terjadi (kecuali telah mendapatkan hasil akhir).