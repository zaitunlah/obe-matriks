\section{Terapan: Solusi persamaan linier multivariabel}
		\paragraph*{} Penggunaan persamaan linier sering terjadi karena karakternya yang dapat mendeskripsikan suatu sistem yang linier. Akan diterapkan metode transformasi matriks untuk mendapatkan solusi penyelesaian persamaan linier. Misalkan diketahui persamaan linier berikut
		
		$$2x_1 + x_2  = 4$$
		$$3x_1 - x_2  = 1$$ 
		
		Persamaan di atas dapat ditulis dalam bentuk matriks seperti berikut
		\begin{align*}
		 \left[
		\begin{array}{rr} 
			2 & 1 \\
			3 & -1 
		\end{array} 
		\right] 
		\left[
		\begin{array}{r} 
			x_1 \\ 
			x_2  
		\end{array}
		\right]
		=
		\left[
		\begin{array}{r} 
			4 \\ 
			1 
		\end{array}
		\right]
		\end{align*}
	apabila memisalkan persamaan tersebut dengan 
	$$\textbf{AX=B}$$
	dengan $A=\left[
	\begin{array}{rr} 
		2 & 1 \\
		3 & -1 
	\end{array} 
	\right]$, $X=\left[
	\begin{array}{r} 
		x_1 \\ 
		x_2  
	\end{array}
	\right]$, dan $B=\left[
	\begin{array}{r} 
		4 \\ 
		1 
	\end{array}
	\right]$ maka akan didapatkan solusi dari persamaan linier tersebut adalah
	\begin{align*}
		\textbf{X}=\textbf{A}^{-1}\textbf{B}
	\end{align*}
	dengan menggunakan invers matriks yang telah didapatkan sebelumnya (Bagian 2) maka
	$$X=\left[
	\begin{array}{rr} 
		1/5 & 1/5 \\
		3/5 & -2/5 
	\end{array} 
	\right]\left[
	\begin{array}{r} 
		4 \\ 
		1 
	\end{array}
	\right]$$
	$$X=\left[
	\begin{array}{r} 
		1 \\ 
		2 
	\end{array}
	\right]$$
	
	
	Namun di sini akan digunakan metode transformasi untuk mendapatkan langsung solusi sistem linier tersebut dengan menuliskan bentuk berikut
	$$A~X~|~~B$$
	proses transformasi ini memiliki tujuan untuk mengubah matriks $A$ menjadi matriks identitas. Secara jelas dapat dilihat pada langkah-langkah berikut
		\begin{align*}
		\begin{array}{r|r} 
			\left[
			\begin{array}{rr} 
				2 & 1 \\
				3 & -1 
			\end{array} 
			\right] \left[
			\begin{array}{r} 
				x_1 \\ 
				x_2  
			\end{array}
			\right]  &  \left[
			\begin{array}{r} 
				4\\
				1 
			\end{array} 
			\right]
		\end{array}
	\end{align*}
	Langkah 1: $\textbf{b}^*_1= \textbf{b}_1 + \textbf{b}_2$ \\
		\begin{align*}
		\begin{array}{r|r} 
			\left[
			\begin{array}{rr} 
				5 & 0 \\
				3 & -1 
			\end{array} 
			\right] \left[
			\begin{array}{r} 
				x_1 \\ 
				x_2  
			\end{array}
			\right]  &  \left[
			\begin{array}{r} 
				5\\
				1 
			\end{array} 
			\right]
		\end{array}
	\end{align*}
	Langkah 2 : $\textbf{b}^*_2= 5\textbf{b}_2 - 3\textbf{b}_1$ \\
		\begin{align*}
		\begin{array}{r|r} 
			\left[
			\begin{array}{rr} 
				5 & 0 \\
				0 & -5 
			\end{array} 
			\right] \left[
			\begin{array}{r} 
				x_1 \\ 
				x_2 
			\end{array}
			\right]  &  \left[
			\begin{array}{r} 
				5\\
				-10 
			\end{array} 
			\right]
		\end{array}
	\end{align*}
	Langkah 3 : $\textbf{b}^*_1= \dfrac{1}{5}\textbf{b}_1 $ dan $\textbf{b}^*_2= -\dfrac{1}{5}\textbf{b}_2$ \\
	\begin{align*}
	\begin{array}{r|r} 
		\left[
		\begin{array}{rr} 
			1 & 0 \\
			0 & 1 
		\end{array} 
		\right] \left[
		\begin{array}{r} 
			x_1 \\ 
			x_2  
		\end{array}
		\right]  &  \left[
		\begin{array}{r} 
			1\\
			2 
		\end{array} 
		\right]
	\end{array}
\end{align*}
	apabila dilihat hasil dari perkalian matriks di atas maka akan didapatkan bahwa $x_1=1$ dan $x_2=2$ yang merupakan solusi dari sistem linier untuk dua variabel.\\
	
	
	Permasalahan berikutnya yaitu untuk sistem dengan tiga variabel seperti berikut
	\begin{align*}
		-2x_1 + x_2  + 2x_3 &= -1 \\
		2x_1 + 3x_2  + x_3 &= 11  \\
		-2x_1 + x_2  + 3x_3 &= 1 
	\end{align*}
sehingga dengan menyusun matriks dari sistem tersebut, didapatkan
	\begin{align*}
		\begin{array}{r|r} 
			\left[
			\begin{array}{rrr} 
				-2 & 1 & 2\\
				2 & 3 & 1\\
				-2 &1 & 3 
			\end{array} 
			\right] 
			\left[
			\begin{array}{r} 
				x_1 \\ 
				x_2 \\
				x_3
			\end{array}
			\right] 
			 & \left[
			\begin{array}{r} 
				-1\\
				11\\
				1 
			\end{array} 
			\right]
		\end{array}
	\end{align*}
	Langkah 1: $\textbf{b}^*_1= \textbf{b}_1 - 2\textbf{b}_2~~,~~\textbf{b}^*_2= 3\textbf{b}_2 - \textbf{b}_3$ \\
	\begin{align*}
		\begin{array}{r|r} 
			\left[
			\begin{array}{rrr} 
				-6 & -5 & 0\\
				8 & 8 & 0\\
				-2 &1 & 3 
			\end{array} 
			\right] 
			\left[
			\begin{array}{r} 
				x_1 \\ 
				x_2 \\
				x_3
			\end{array}
			\right]
			 & \left[
			 \begin{array}{r} 
			 	-23\\
			 	32\\
			 	1 
			 \end{array} 
			 \right]
		\end{array}
	\end{align*}
	Langkah 2 : $\textbf{b}^*_1= 8\textbf{b}_1 + 5\textbf{b}_2~~,~~\textbf{b}^*_3= 8\textbf{b}_3 - \textbf{b}_2$ \\
	\begin{align*}
		\begin{array}{r|r} 
			\left[
			\begin{array}{rrr} 
				-8 & 0 & 0\\
				8 & 8 & 0\\
				-24 &0 & 24 
			\end{array} 
			\right] 
			\left[
			\begin{array}{r} 
				x_1 \\ 
				x_2 \\
				x_3
			\end{array}
			\right]  & \left[
			\begin{array}{r} 
				-24\\
				32\\
				-24 
			\end{array} 
			\right]
		\end{array}
	\end{align*}
	Langkah 4 : $\textbf{b}^*_1= -\dfrac{1}{8}\textbf{b}_1~~,~~\textbf{b}^*_2= \dfrac{1}{8}\textbf{b}_2~~,~~\textbf{b}^*_3= \dfrac{1}{24}\textbf{b}_3$ \\
	\begin{align*}
		\begin{array}{r|r} 
			\left[
			\begin{array}{rrr} 
				1 & 0 & 0\\
				1 & 1 & 0\\
				-1 &0 & 1 
			\end{array} 
			\right]\left[
			\begin{array}{r} 
				x_1 \\ 
				x_2 \\
				x_3
			\end{array}
			\right]  & \left[
			\begin{array}{r} 
				3\\
				4\\
				-1 
			\end{array} 
			\right]
		\end{array}
	\end{align*}
	Langkah 5 : $\textbf{b}^*_2= \textbf{b}_2 - \textbf{b}_1~~,~~\textbf{b}^*_3= \textbf{b}_3 + \textbf{b}_1$ \\
	\begin{align*}
		\begin{array}{r|r} 
			\left[
			\begin{array}{rrr} 
				1 & 0 & 0\\
				0 & 1 & 0\\
				0 &0 & 1 
			\end{array} 
			\right] \left[
			\begin{array}{r} 
				x_1 \\ 
				x_2 \\
				x_3
			\end{array}
			\right] & \left[
			\begin{array}{r} 
				3\\
				1\\
				2 
			\end{array} 
			\right]
		\end{array}
	\end{align*}
didapatkan solusi sistem linier yaitu $x_1=3$, $x_2=1$, dan $x_3=2$. ($\bigstar$ Buktikan dengan $X=A^{-1}B$) \\

Perlu diketahui bahwa metode penyelesaian sistem linier seperti ini (transformasi baris) dengan tujuan mengubah matriks di bagian kiri menjadi matriks identitas merupakan Metode Gauss-Jordan.