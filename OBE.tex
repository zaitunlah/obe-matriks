\documentclass{article}   % Specifies the document class
\usepackage{amsmath}  % For math
\usepackage{amssymb}  % For more math
\usepackage{enumerate} % useful for itemization
\usepackage{enumitem}
\usepackage{siunitx}  % standardization of si units
\usepackage{graphicx}
%Stop indentation on new paragraphs
\usepackage[parfill]{parskip}
\usepackage{lastpage}
\usepackage{setspace}
\usepackage{geometry}
\geometry{
	a4paper,
	%total={170mm,257mm},
	left=25mm,
	right=20mm,
	top=25mm,
	bottom=20mm,
}
\newcommand{\judul}{Transformasi Matriks: Operasi Baris untuk Persoalan Sistem Linier}
\newcommand{\topik}{Matematika Komputasi}

%============== For header and footer
\usepackage{fancyhdr}
\pagestyle{fancy}
%  --------   Normal Headers
\fancyhf{} % clear all fields
%\fancyhead[C]{Paper Title: Modeling Reliability of the Grid}
\fancyfoot[L]{Copyright \textcopyright 2014 MECS}
\fancyfoot[C]{\textsf{\thepage ~of~ \pageref{LastPage}}}
\fancyfoot[R]{\textsf{2014, vol, issue, pages}}
\renewcommand{\headrulewidth}{0pt} % to remove line on header
\renewcommand{\footrulewidth}{0pt} % to remove line on footer

% ----------------------------define new page style for first page
% \fancypagestyle{first}{%
% 	\fancyhf{}% clear all header and footer fields
% 	\fancyhead[L]{\JournalDataAndLogo}%
% 	\fancyfoot[L]{Copyright \textcopyright 2014 MECS}%
% 	\fancyfoot[R]{\emph{I.J. Computer Network and Information Security}, 2014, vol, issue, page}%
% 	\renewcommand{\headrulewidth}{0pt}% to remove line on header
% 	\renewcommand{\footrulewidth}{0pt}% to remove line on footer
% }

%\newcommand{\@judul}{Judul T.A.}
%\DeclareRobustCommand{\Judul}[1]{%
%	\renewcommand{\@judul}{#1}%
% \newcommand{\JournalDataAndLogo}{%
% 	\leavevmode\smash{%
% 		\raisebox{-2ex}{% <----- adjust to suit
% 			\begin{tabular*}{\textwidth}{@{\extracolsep{\fill}}lr@{}}
% 				\begin{tabular}[b]{@{}l@{}}
% 					\emph{\textbf{I.J. Computer Network and Information Security}}, 2014, Vol, issue, pages\\
% 					Published Online Month 2014 in MECS (http://www.mecs-press.org/)\\
% 					DOI:
% 				\end{tabular}
% 				&
% 				\includegraphics[width=2.5cm,height=1.5cm]{example-image.pdf}
% % 			\end{tabular*}%
% 		}% end of \raisebox
% 	}% end of \smash
% }
% %==============

\begin{document}
	
\begin{center}
    \vspace{.4cm}
    \textsf{\textbf { \large \judul}}
\end{center}
\vspace{.4cm}
\hrule
    \textsf{\\
    \textbf{Zaitun,S.Si.,M.Mat.} \hspace{\fill}
    \textit{\large }\\ [0.7ex]
    \textit{zet.zaitun@gmail.com} \hspace{\fill} \textbf{}\\ [0.7ex]
    Institut Teknologi Bacharuddin Jusuf Habibie \hspace{\fill} \textbf{-\topik}} \\
\hrule
\vspace{.4cm}
%============================MULAI DOKUMEN
\begin{center}
\textbf{\Large Motivasi}
\end{center}
\paragraph*{} 
Transformasi baris atau operasi baris dikenal sebagai proses baris elementer. Transformasi ini tidak mengubah struktur dari matriks tetapi hanya mengubah elemen dari matriks. Penggunaan transformasi ini akan memudahkan dalam menyelesaikan persoalan matriks. Beberapa persoalan matriks seperti menentukan invers matriks akan diterapkan operasi baris, kemudian pada akhir akan digunakan pula untuk menentukan solusi dari sistem persamaan linier multivariabel. Dengan adanya tulisan ini harapannya dapat membantu memahami proses dan langkah penyelesaian matriks dengan transformasi.
%section
		\section{Transformasi Baris}
		\paragraph*{} Misalkan matriks $A$ berordo $2\times2$ akan dilakukan transformasi untuk mendapatkan matriks baru dengan nilai elemen diagonalnya bernilai $1$ atau dengan kata lain menjadikan matriks $A$ menjadi matriks identitas. Penggunaan $\textbf{b}$ akan digunakan sebagai simbol baris dan \textit{superscript} ($*$) sebagai tanda update nilai baris pada transformasi ini.
		\begin{align*}
		A = \left[
		\begin{array}{rr} 
			2 & 1 \\
			3 & -1 
		\end{array} 
		\right]
		\end{align*}
		Langkah 1: $\textbf{b}^*_1= \textbf{b}_1 + \textbf{b}_2$ \\
		\begin{align*}
			A=\left[
			\begin{array}{rr} 
				5 & 0 \\
				3 & -1 
			\end{array} 
			\right]
		\end{align*}
		Langkah 2 : $\textbf{b}^*_2= 5\textbf{b}_2 - 3\textbf{b}_1$ \\
		\begin{align*}
			\left[
			\begin{array}{rr} 
				5 & 0 \\
				0 & -5 
			\end{array} 
			\right]
		\end{align*}
		Langkah 3 : $\textbf{b}^*_1= \dfrac{1}{5}\textbf{b}_1 $ dan $\textbf{b}^*_2= -\dfrac{1}{5}\textbf{b}_2$ \\
		\begin{align*}
			\left[
			\begin{array}{rr} 
				1 & 0 \\
				0 & 1
			\end{array} 
			\right]
		\end{align*}
		dari hasil di atas, tujuan mendapatkan matriks identitas tercapai. 
		
		\subsection[]{Determinan}Hal yang cukup menarik pada transformasi baris adalah jika diperhatikan pada langkah 1, dapat kita hitung minor baris 2 kolom 2 $M_{22}$ kemudian menghitung kofaktor baris 2 kolom 2 $K_{22}$ selanjutnya akan didapat nilai determinan dari matriks tersebut
		$$K_{ij} = (-1)^{i+j}M_{ij}$$
		\begin{align*}
			K_{22} &= (-1)^{4}M_{22}\\
			K_{22} &= 5
		\end{align*}
		dengan
		$$det(A)=\sum_{i=1}^{n} A_{ij}K_{ij}~~~~ \texttt{untuk $j$ tetap}$$
		atau 
		$$det(A)=\sum_{j=1}^{n} A_{ij}K_{ij}~~~~ \texttt{untuk $i$ tetap}$$
		karena dipakai kolom $2$ sehingga $j=2$ tetap
		\begin{align*}
			det(A)&=A_{12}K_{12} + A_{22}K_{22}\\
			det(A)&=(0)K_{12} + (-1)(5)\\
			det(A)&=-5
		\end{align*}
			
		\subsection{Matriks $3\times3$}
		% Transformasi ini tidak membatasi ordo matriks, sehingga untuk ordo berapapun dapat diselesaikan. Tambahan bahwa untuk ukuran matriks yang lebih besar, dapat diterapkan dua operasi sekaligus agar langkah proses lebih efisien. Misalkan matriks ordo $3\times3$ berikut
				\begin{align*}
		\left[
		\begin{array}{rrr} 
			-2 & 1 & 2\\
			2 & 3 & 1\\
			-2 &1 & 3 
		\end{array} 
		\right]
	\end{align*}
Langkah 1: $\textbf{b}^*_1= \textbf{b}_1 - 2\textbf{b}_2~~,~~\textbf{b}^*_2= 3\textbf{b}_2 - \textbf{b}_3$ \\
\begin{align*}
\left[
\begin{array}{rrr} 
	-6 & -5 & 0\\
	8 & 8 & 0\\
	-2 &1 & 3 
\end{array} 
\right]
\end{align*}
Langkah 2 : $\textbf{b}^*_2= \dfrac{1}{8}\textbf{b}_2$ \\
\begin{align*}
\left[
\begin{array}{rrr} 
	-6 & -5 & 0\\
	1 & 1 & 0\\
	-2 &1 & 3 
\end{array} 
\right]
\end{align*}
Langkah 3 : $\textbf{b}^*_1= \textbf{b}_1 + 5\textbf{b}_2~~,~~\textbf{b}^*_3= \textbf{b}_3 - \textbf{b}_2$ \\
\begin{align*}
\left[
\begin{array}{rrr} 
	-1 & 0 & 0\\
	1 & 1 & 0\\
	-3 &0 & 3 
\end{array} 
\right]
\end{align*}
Langkah 4 : $\textbf{b}^*_1= -\textbf{b}_1~~,~~\textbf{b}^*_3= \dfrac{1}{3}\textbf{b}_3$ \\
\begin{align*}
	\left[
	\begin{array}{rrr} 
		1 & 0 & 0\\
		1 & 1 & 0\\
		-1 &0 & 1 
	\end{array} 
	\right]
\end{align*}
Langkah 5 : $\textbf{b}^*_2= \textbf{b}_2 - \textbf{b}_1~~,~~\textbf{b}^*_3= \textbf{b}_3 + \textbf{b}_1$ \\
\begin{align*}
	\left[
	\begin{array}{rrr} 
		1 & 0 & 0\\
		0 & 1 & 0\\
		0 &0 & 1 
	\end{array} 
	\right]
\end{align*}
penting untuk diketahui bahwa proses transformasi perlu diperhatikan saat melakukan operasi untuk tidak menggunakan baris yang telah diupdate dimasukkan ke dalam operasi lain dalam satu langkah proses.
\section{Invers Matriks}
		\paragraph*{} Misalkan matriks $A$ memiliki invers $A^{-1}$, pada proses mendapatkan invers tersebut maka transformasi matriks yang telah dijelaskan sebelumnya akan digunakan untuk mendapatkan invers matriks dengan bentuk 
		$$A~ \vdots~  I~~~~ \longrightarrow ~~~~I ~\vdots~ A^{-1}$$
		dengan menuliskan matriks identitas di bagian kanan matriks $A$ kemudian operasi baris dilakukan bersama dengan matriks identitas.
		\subsection{Matriks $2\times 2$}
		Misalkan diketahui matriks 
		\begin{align*}
		A = \left[
		\begin{array}{rr} 
			2 & 1 \\
			3 & -1 
		\end{array} 
		\right]
	\end{align*}
maka untuk menentukan invers matriks $A$ adalah dengan menuliskan matriks identitas di sebelah kanan
 	\begin{align*}
 	\begin{array}{r|r} 
 		 \left[
 		\begin{array}{rr} 
 			2 & 1 \\
 			3 & -1 
 		\end{array} 
 		\right]  &  \left[
 		\begin{array}{rr} 
 			1 & 0 \\
 			0 & 1 
 		\end{array} 
 		\right]
 	\end{array}
 \end{align*}
Langkah 1: $\textbf{b}^*_1= \textbf{b}_1 + \textbf{b}_2$ \\
\begin{align*}
	\begin{array}{r|r} 
		\left[
		\begin{array}{rr} 
			5 & 0 \\
			3 & -1 
		\end{array} 
		\right]  &  \left[
		\begin{array}{rr} 
			1 & 1 \\
			0 & 1 
		\end{array} 
		\right]
	\end{array}
\end{align*}
Langkah 2 : $\textbf{b}^*_2= 5\textbf{b}_2 - 3\textbf{b}_1$ \\
\begin{align*}
	\begin{array}{r|r} 
		\left[
		\begin{array}{rr} 
			5 & 0 \\
			0 & -5 
		\end{array} 
		\right]  &  \left[
		\begin{array}{rr} 
			1 & 1 \\
			-3 & 2 
		\end{array} 
		\right]
	\end{array}
\end{align*}
Langkah 3 : $\textbf{b}^*_1= \dfrac{1}{5}\textbf{b}_1 $ dan $\textbf{b}^*_2= -\dfrac{1}{5}\textbf{b}_2$ \\
\begin{align*}
	\begin{array}{r|r} 
		\left[
		\begin{array}{rr} 
			1 & 0 \\
			0 & 1 
		\end{array} 
		\right]  &  \left[
		\begin{array}{rr} 
			1/5 & 1/5 \\
			3/5 & -2/5 
		\end{array} 
		\right]
	\end{array}
\end{align*}
dari hasil di atas, tujuan mendapatkan invers matriks  tercapai sesuai $I ~\vdots~ A^{-1}$ sehingga $$A^{-1}=\left[
\begin{array}{rr} 
	1/5 & 1/5 \\
	3/5 & -2/5 
\end{array} 
\right]$$
\subsection{Matriks $3\times 3$}
Serupa dengan proses sebelumnya, untuk ukuran matriks yang lebih besar metode transformasi akan sangat membantu untuk mendapatkan invers matriks. Misalkan matriks berordo $3\times3$
\begin{align*}
	\begin{array}{r|r} 
		\left[
	\begin{array}{rrr} 
		-2 & 1 & 2\\
		2 & 3 & 1\\
		-2 &1 & 3 
	\end{array} 
	\right]  & \left[
	\begin{array}{rrr} 
		1 & 0 & 0\\
		0 & 1 & 0\\
		0 &0 & 1 
	\end{array} 
	\right]
	\end{array}
\end{align*}
Langkah 1: $\textbf{b}^*_1= \textbf{b}_1 - 2\textbf{b}_2~~,~~\textbf{b}^*_2= 3\textbf{b}_2 - \textbf{b}_3$ \\
\begin{align*}
	\begin{array}{r|r} 
		\left[
		\begin{array}{rrr} 
			-6 & -5 & 0\\
			8 & 8 & 0\\
			-2 &1 & 3 
		\end{array} 
		\right]  & \left[
		\begin{array}{rrr} 
			1 & -2 & 0\\
			0 & 3 & -1\\
			0 &0 & 1 
		\end{array} 
		\right]
	\end{array}
\end{align*}
Langkah 2 : $\textbf{b}^*_1= 8\textbf{b}_1 + 5\textbf{b}_2~~,~~\textbf{b}^*_3= 8\textbf{b}_3 - \textbf{b}_2$ \\
\begin{align*}
	\begin{array}{r|r} 
		\left[
		\begin{array}{rrr} 
			-8 & 0 & 0\\
			8 & 8 & 0\\
			-24 &0 & 24 
		\end{array} 
		\right]  & \left[
		\begin{array}{rrr} 
			8 & -1 & -5\\
			0 & 3 & -1\\
			0 &-3 & 9 
		\end{array} 
		\right]
	\end{array}
\end{align*}
Langkah 3 : $\textbf{b}^*_3= \dfrac{1}{3}\textbf{b}_3$ \\
\begin{align*}
	\begin{array}{r|r} 
		\left[
		\begin{array}{rrr} 
			-8 & 0 & 0\\
			8 & 8 & 0\\
			-8 &0 & 8 
		\end{array} 
		\right]  & \left[
		\begin{array}{rrr} 
			8 & -1 & -5\\
			0 & 3 & -1\\
			0 &-1 & 3 
		\end{array} 
		\right]
	\end{array}
\end{align*}
Langkah 4 : $\textbf{b}^*_2= \textbf{b}_2 + \textbf{b}_1~~,~~\textbf{b}^*_3= \textbf{b}_3 - \textbf{b}_1$ \\
\begin{align*}
	\begin{array}{r|r} 
		\left[
		\begin{array}{rrr} 
			-8 & 0 & 0\\
			0 & 8 & 0\\
			0 &0 & 8 
		\end{array} 
		\right]  & \left[
		\begin{array}{rrr} 
			8 & -1 & -5\\
			8 & 2 & -6\\
			-8 &0 & 8 
		\end{array} 
		\right]
	\end{array}
\end{align*}
Langkah 5 : $\textbf{b}^*_1= -\dfrac{1}{8}\textbf{b}_1~~,~~\textbf{b}^*_2= \dfrac{1}{8}\textbf{b}_2~~,~~\textbf{b}^*_3= \dfrac{1}{8}\textbf{b}_3$ \\
\begin{align*}
	\begin{array}{r|r} 
		\left[
		\begin{array}{rrr} 
			1 & 0 & 0\\
			0 & 1 & 0\\
			0 &0 & 1 
		\end{array} 
		\right]  & \left[
		\begin{array}{rrr} 
			-1 & 1/8 & 5/8\\
			1 & 1/4 & -3/4\\
			-1 &0 & 1 
		\end{array} 
		\right]
	\end{array}
\end{align*}
perlu namun tidak harus yaitu saat melakukan operasi kalau bisa tidak menggunakan bilangan pecahan untuk mengantisipasi kesalahan perhitungan yang kadang terjadi (kecuali telah mendapatkan hasil akhir).
\section{Terapan: Solusi persamaan linier multivariabel}
		\paragraph*{} Penggunaan persamaan linier sering terjadi karena karakternya yang dapat mendeskripsikan suatu sistem yang linier. Akan diterapkan metode transformasi matriks untuk mendapatkan solusi penyelesaian persamaan linier. Misalkan diketahui persamaan linier berikut
		
		$$2x_1 + x_2  = 4$$
		$$3x_1 - x_2  = 1$$ 
		
		Persamaan di atas dapat ditulis dalam bentuk matriks seperti berikut
		\begin{align*}
		 \left[
		\begin{array}{rr} 
			2 & 1 \\
			3 & -1 
		\end{array} 
		\right] 
		\left[
		\begin{array}{r} 
			x_1 \\ 
			x_2  
		\end{array}
		\right]
		=
		\left[
		\begin{array}{r} 
			4 \\ 
			1 
		\end{array}
		\right]
		\end{align*}
	apabila memisalkan persamaan tersebut dengan 
	$$\textbf{AX=B}$$
	dengan $A=\left[
	\begin{array}{rr} 
		2 & 1 \\
		3 & -1 
	\end{array} 
	\right]$, $X=\left[
	\begin{array}{r} 
		x_1 \\ 
		x_2  
	\end{array}
	\right]$, dan $B=\left[
	\begin{array}{r} 
		4 \\ 
		1 
	\end{array}
	\right]$ maka akan didapatkan solusi dari persamaan linier tersebut adalah
	\begin{align*}
		\textbf{X}=\textbf{A}^{-1}\textbf{B}
	\end{align*}
	dengan menggunakan invers matriks yang telah didapatkan sebelumnya (Bagian 2) maka
	$$X=\left[
	\begin{array}{rr} 
		1/5 & 1/5 \\
		3/5 & -2/5 
	\end{array} 
	\right]\left[
	\begin{array}{r} 
		4 \\ 
		1 
	\end{array}
	\right]$$
	$$X=\left[
	\begin{array}{r} 
		1 \\ 
		2 
	\end{array}
	\right]$$
	
	
	Namun di sini akan digunakan metode transformasi untuk mendapatkan langsung solusi sistem linier tersebut dengan menuliskan bentuk berikut
	$$A~X~|~~B$$
	proses transformasi ini memiliki tujuan untuk mengubah matriks $A$ menjadi matriks identitas. Secara jelas dapat dilihat pada langkah-langkah berikut
		\begin{align*}
		\begin{array}{r|r} 
			\left[
			\begin{array}{rr} 
				2 & 1 \\
				3 & -1 
			\end{array} 
			\right] \left[
			\begin{array}{r} 
				x_1 \\ 
				x_2  
			\end{array}
			\right]  &  \left[
			\begin{array}{r} 
				4\\
				1 
			\end{array} 
			\right]
		\end{array}
	\end{align*}
	Langkah 1: $\textbf{b}^*_1= \textbf{b}_1 + \textbf{b}_2$ \\
		\begin{align*}
		\begin{array}{r|r} 
			\left[
			\begin{array}{rr} 
				5 & 0 \\
				3 & -1 
			\end{array} 
			\right] \left[
			\begin{array}{r} 
				x_1 \\ 
				x_2  
			\end{array}
			\right]  &  \left[
			\begin{array}{r} 
				5\\
				1 
			\end{array} 
			\right]
		\end{array}
	\end{align*}
	Langkah 2 : $\textbf{b}^*_2= 5\textbf{b}_2 - 3\textbf{b}_1$ \\
		\begin{align*}
		\begin{array}{r|r} 
			\left[
			\begin{array}{rr} 
				5 & 0 \\
				0 & -5 
			\end{array} 
			\right] \left[
			\begin{array}{r} 
				x_1 \\ 
				x_2 
			\end{array}
			\right]  &  \left[
			\begin{array}{r} 
				5\\
				-10 
			\end{array} 
			\right]
		\end{array}
	\end{align*}
	Langkah 3 : $\textbf{b}^*_1= \dfrac{1}{5}\textbf{b}_1 $ dan $\textbf{b}^*_2= -\dfrac{1}{5}\textbf{b}_2$ \\
	\begin{align*}
	\begin{array}{r|r} 
		\left[
		\begin{array}{rr} 
			1 & 0 \\
			0 & 1 
		\end{array} 
		\right] \left[
		\begin{array}{r} 
			x_1 \\ 
			x_2  
		\end{array}
		\right]  &  \left[
		\begin{array}{r} 
			1\\
			2 
		\end{array} 
		\right]
	\end{array}
\end{align*}
	apabila dilihat hasil dari perkalian matriks di atas maka akan didapatkan bahwa $x_1=1$ dan $x_2=2$ yang merupakan solusi dari sistem linier untuk dua variabel.\\
	
	
	Permasalahan berikutnya yaitu untuk sistem dengan tiga variabel seperti berikut
	\begin{align*}
		-2x_1 + x_2  + 2x_3 &= -1 \\
		2x_1 + 3x_2  + x_3 &= 11  \\
		-2x_1 + x_2  + 3x_3 &= 1 
	\end{align*}
sehingga dengan menyusun matriks dari sistem tersebut, didapatkan
	\begin{align*}
		\begin{array}{r|r} 
			\left[
			\begin{array}{rrr} 
				-2 & 1 & 2\\
				2 & 3 & 1\\
				-2 &1 & 3 
			\end{array} 
			\right] 
			\left[
			\begin{array}{r} 
				x_1 \\ 
				x_2 \\
				x_3
			\end{array}
			\right] 
			 & \left[
			\begin{array}{r} 
				-1\\
				11\\
				1 
			\end{array} 
			\right]
		\end{array}
	\end{align*}
	Langkah 1: $\textbf{b}^*_1= \textbf{b}_1 - 2\textbf{b}_2~~,~~\textbf{b}^*_2= 3\textbf{b}_2 - \textbf{b}_3$ \\
	\begin{align*}
		\begin{array}{r|r} 
			\left[
			\begin{array}{rrr} 
				-6 & -5 & 0\\
				8 & 8 & 0\\
				-2 &1 & 3 
			\end{array} 
			\right] 
			\left[
			\begin{array}{r} 
				x_1 \\ 
				x_2 \\
				x_3
			\end{array}
			\right]
			 & \left[
			 \begin{array}{r} 
			 	-23\\
			 	32\\
			 	1 
			 \end{array} 
			 \right]
		\end{array}
	\end{align*}
	Langkah 2 : $\textbf{b}^*_1= 8\textbf{b}_1 + 5\textbf{b}_2~~,~~\textbf{b}^*_3= 8\textbf{b}_3 - \textbf{b}_2$ \\
	\begin{align*}
		\begin{array}{r|r} 
			\left[
			\begin{array}{rrr} 
				-8 & 0 & 0\\
				8 & 8 & 0\\
				-24 &0 & 24 
			\end{array} 
			\right] 
			\left[
			\begin{array}{r} 
				x_1 \\ 
				x_2 \\
				x_3
			\end{array}
			\right]  & \left[
			\begin{array}{r} 
				-24\\
				32\\
				-24 
			\end{array} 
			\right]
		\end{array}
	\end{align*}
	Langkah 4 : $\textbf{b}^*_1= -\dfrac{1}{8}\textbf{b}_1~~,~~\textbf{b}^*_2= \dfrac{1}{8}\textbf{b}_2~~,~~\textbf{b}^*_3= \dfrac{1}{24}\textbf{b}_3$ \\
	\begin{align*}
		\begin{array}{r|r} 
			\left[
			\begin{array}{rrr} 
				1 & 0 & 0\\
				1 & 1 & 0\\
				-1 &0 & 1 
			\end{array} 
			\right]\left[
			\begin{array}{r} 
				x_1 \\ 
				x_2 \\
				x_3
			\end{array}
			\right]  & \left[
			\begin{array}{r} 
				3\\
				4\\
				-1 
			\end{array} 
			\right]
		\end{array}
	\end{align*}
	Langkah 5 : $\textbf{b}^*_2= \textbf{b}_2 - \textbf{b}_1~~,~~\textbf{b}^*_3= \textbf{b}_3 + \textbf{b}_1$ \\
	\begin{align*}
		\begin{array}{r|r} 
			\left[
			\begin{array}{rrr} 
				1 & 0 & 0\\
				0 & 1 & 0\\
				0 &0 & 1 
			\end{array} 
			\right] \left[
			\begin{array}{r} 
				x_1 \\ 
				x_2 \\
				x_3
			\end{array}
			\right] & \left[
			\begin{array}{r} 
				3\\
				1\\
				2 
			\end{array} 
			\right]
		\end{array}
	\end{align*}
didapatkan solusi sistem linier yaitu $x_1=3$, $x_2=1$, dan $x_3=2$. ($\bigstar$ Buktikan dengan $X=A^{-1}B$) \\

Perlu diketahui bahwa metode penyelesaian sistem linier seperti ini (transformasi baris) dengan tujuan mengubah matriks di bagian kiri menjadi matriks identitas merupakan Metode Gauss-Jordan.

\section{Kesimpulan}
\paragraph*{} Penggunaan metode transformasi sangat memudahkan dalam menyelesaikan persoalan matriks karena sifatnya yang fleksibel pada ukuran matriks yang lebih besar. 

\section{Tugas Mandiri}
		\begin{enumerate}
		\item Ubah matriks berikut menjadi matriks identitas menggunakan transformasi matriks
		$$
		\begin{array}{llllllllll} 
			a. & 	A=\left[
			\begin{array}{rr} 
				4 & 1\\
				-1 &1 
			\end{array} 
			\right]
			& b. & B=	\left[
			\begin{array}{rr} 
				-1 & 2\\
				3  & 4 
			\end{array} 
			\right]
			& c. & C=	\left[
			\begin{array}{rrr} 
				1 & 2 & -1\\
				2 & -1 & 3\\
				4 & -2 & 1 
			\end{array} 
			\right]
			& d. &  D=\left[
			\begin{array}{rrr} 
				2 & 4 & 3\\
				3 & 1 & 0\\
				-2 &1 & -2 
			\end{array} 
			\right]
			\\
		\end{array} 
		$$
		\item Hitung invers dari matriks berikut menggunakan transformasi matriks%2
		$$
				\begin{array}{llllllllll} 
			a. & 	A=\left[
			\begin{array}{rr} 
				4 & 1\\
				-1 &1 
			\end{array} 
			\right]
			& b. & B=	\left[
			\begin{array}{rr} 
				-1 & 2\\
				3  & 4 
			\end{array} 
			\right]
			& c. & C=	\left[
			\begin{array}{rrr} 
				1 & 2 & -1\\
				2 & -1 & 3\\
				4 & -2 & 1 
			\end{array} 
			\right]
			& d. &  D=\left[
			\begin{array}{rrr} 
				2 & 4 & 3\\
				3 & 1 & 0\\
				-2 &1 & -2 
			\end{array} 
			\right]
			\\
		\end{array} 
		$$
		\item Selesaikan sistem linier berikut dengan menggunakan transformasi matriks\\ %3
				$$
		\begin{array}{lclc} 
			a. & 
			\begin{array}{rcr} 
				4x_1 + x_2&=& 11 \\
				-x_1 + x_2&=& 1 
			\end{array} ~~~~~~~~~~~~
			& b. & 
			\begin{array}{rcr} 
				-x_1 + 2x_2&=& -1 \\
				3x_1 + 4x_2&=& 13 
			\end{array} \\
		\\
			c. & \begin{array}{rcr} 
				x_1 + 2x_2  - x_3 &=& 4 \\
				2x_1 - x_2  + 3x_3 &=& 3  \\
				4x_1 - 2x_2  + x_3 &=& 1 
			\end{array} ~~~~~~~~~~~~
			& d. &  \begin{array}{lcr} 
				2x_1 + 4x_2  + 3x_3 &=& 13 \\
				3x_1 + x_2  &=& 10  \\
				-2x_1 + x_2  - 2x_3 &=& -7 
			\end{array}\\
			\\
		\end{array}
		$$
		\textbf{\textit{Jawaban:}} a. $(2,3)$, b.$(3,1)$, c.$(1,2,1)$, d. $(3,1,1)$
		\end{enumerate}

\begin{thebibliography}{2}
\bibitem{}Zaitun,				                   %Authors
	\newblock  (2023).					       % Years
	\newblock TULISAN PRIBADI.			       % Title
	\newblock \emph{Journal Name} number issue(section),  % Jurnal name and volume
	\newblock City;										  % City
	\newblock Page number.								  % Pages
\end{thebibliography}
\end{document}